\documentclass[unicode,12pt,aspectratio=169]{beamer}
\usepackage{bxdpx-beamer}
\usetheme[progressbar=frametitle]{metropolis}
\usepackage{zxjatype}
\setCJKmainfont{Noto Sans CJK JP}
\usepackage{bm}
\usepackage{color}
\usepackage{listings,jlisting}
\usepackage{eulervm}
\usepackage{graphicx}
\usepackage{tcolorbox}
\usepackage{amsthm}
\lstset{language={C}, basicstyle=\ttfamily\footnotesize,
commentstyle=\textit, classoffset=1, frame=tRBl, framesep=5pt,
numbers=left, stepnumber=1, numberstyle=\footnotesize, tabsize=2 }
\usepackage{hyperref}
\setsansfont[
BoldFont={Fira Sans SemiBold},
ItalicFont={Fira Sans Italic},
BoldItalicFont={Fira Sans SemiBold Italic}
]{Fira Sans}
\setbeamertemplate{theorems}[]
\newtheorem*{remark}{Remark}
\begin{document}
\begin{frame}{部分圏 P8}
    \begin{tcolorbox}
        \begin{definition}
            圏Dの対象及び射が圏Cに含まれるとき、DをCの部分圏という。
        \end{definition}
        \begin{theorem}
            すべての圏Cはその部分圏にCのすべての同型射を含む最大部分圏が存在する
        \end{theorem}
    \end{tcolorbox}
\end{frame}
\begin{frame}{証明 Excercise 1.1.ii}
    \begin{tcolorbox}
        \begin{remark}
            ここで、ツォルンの補題等集合論の公理や諸定理を用いてはならない
        \end{remark}
        \begin{proof}
            (存在)\\
            すべての対象に対してその自己同型射は同型である。また、すべての対象及びその自己同型射からなる集まりは部分圏となる。よりこれは亜群となる。

            (すべての同型射からなる集まりは圏となる)\\
            ホワイトボードでやります・・・
        \end{proof}
    \end{tcolorbox}
\end{frame}
\begin{frame}{証明 Excercise 1.1.iii}
    \begin{tcolorbox}
        ホワイトボードにて・・・
    \end{tcolorbox}
\end{frame}
\begin{frame}{反対圏 P9,10}
    \begin{tcolorbox}
        \begin{definition}
            圏Cに対して、同じ対象をもち、すべての射に対して始域と終域を入れ替えた射を持っている圏$C^{op}$を反対圏という
        \end{definition}
        これにより、圏の定理は二重性をもつ。例示は以下にする。
        \begin{theorem}
            以下の3つは同値である。
            \begin{enumerate}
                \item 射$f:x \rightarrow y$が同型
                \item すべての対象$c$に対して$f_*:C(c,x) \rightarrow C(c,y)$が全単射
                \item すべての対象$c$に対して$f^*:C(x,c) \rightarrow C(y,c)$が全単射
            \end{enumerate}
        \end{theorem}
    \end{tcolorbox}
\end{frame}
\end{document}
