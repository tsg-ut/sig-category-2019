\documentclass[unicode,12pt,aspectratio=169]{beamer}
\usepackage{bxdpx-beamer}
\usetheme[progressbar=frametitle]{metropolis}
\usepackage{zxjatype}
\setCJKmainfont{Noto Sans CJK JP}
\usepackage{bm}
\usepackage{color}
\usepackage{listings,jlisting}
\usepackage{eulervm}
\usepackage{graphicx}
\usepackage{tcolorbox}
\usepackage{amsthm}
\lstset{language={C}, basicstyle=\ttfamily\footnotesize,
commentstyle=\textit, classoffset=1, frame=tRBl, framesep=5pt,
numbers=left, stepnumber=1, numberstyle=\footnotesize, tabsize=2 }
\usepackage{hyperref}
\setsansfont[
BoldFont={Fira Sans SemiBold},
ItalicFont={Fira Sans Italic},
BoldItalicFont={Fira Sans SemiBold Italic}
]{Fira Sans}
\setbeamertemplate{theorems}[]
\newtheorem*{remark}{Remark}
\begin{document}
\begin{frame}{モノ射とエピ射 P11}
    \begin{tcolorbox}
       \begin{definition}
           \begin{enumerate}
               \item モノ射または単射とは$f\circ g=f\circ h \Rightarrow g=h$であるような$f$である。即ち、$f_*$が単射である。
               \item エピ射または全射とは$g\circ f=h\circ f \Rightarrow g=h$であるような$f$である。即ち、$f^*$が単射である。
           \end{enumerate}
       \end{definition}
    \end{tcolorbox}
\end{frame}
\begin{frame}{注意点P12}
    \begin{tcolorbox}
        \begin{remark}
            圏の射が何かしらの写像だった場合、集合論的全射であればモノ射であり、集合論的単射であればエピ射であるが、逆は必ずしも成り立たない。環の圏において包含環準同型$\mathbb{Z}\hookrightarrow \mathbb{Q}$は全単射ではないがモノ射でありエピ射である。
        \end{remark}
    \end{tcolorbox}
\end{frame}
\begin{frame}{補題1.2.11}
    \begin{tcolorbox}
        \begin{theorem}
            \begin{enumerate}
                \item モノ射の合成はモノ射
                \item モノ射$gf$の$f$はモノ射
                \item エピ射の合成はエピ射
                \item エピ射$gf$の$g$はエピ射
            \end{enumerate}
            \begin{proof}
                ホワイトボードにて・・・
            \end{proof}
        \end{theorem}
    \end{tcolorbox}
\end{frame}
\begin{frame}{関手}
    \begin{tcolorbox}
        \begin{definition}
            関手$F:C \rightarrow D$は
            \begin{enumerate}
                \item $C$の対象$c$を$D$の対象$F c$に移す
                \item $C$の射$f:A\rightarrow B$をDの射$F f:F A \rightarrow F B$に移す
                \item $C$の恒等射は$D$の恒等射に移す
                \item $F g\cdot F f=F(g \circ f)$
            \end{enumerate}
        \end{definition}
    \end{tcolorbox}
\end{frame}
\begin{frame}{位相空間での使用例P15}
    \begin{tcolorbox}
        \begin{theorem}
            $f:D^2 \rightarrow D^2$な連続関数は必ず不動点を持つ
        \end{theorem}
        \begin{proof}
            ホワイトボードにて・・・
        \end{proof}
    \end{tcolorbox}
\end{frame}
\begin{frame}{共変と反変}
    \begin{tcolorbox}
        \begin{definition}
            先程定義したのは共変関手である。また、圏$C^op$と$D$を対応させる場合を反変関手という。
        \end{definition}
    \end{tcolorbox}
\end{frame}
\begin{frame}{}
    \begin{tcolorbox}
    \end{tcolorbox}
\end{frame}
\end{document}
